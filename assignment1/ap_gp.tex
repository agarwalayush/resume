\documentclass[a4paper,10pt]{article}
\usepackage[utf8]{inputenc}
\usepackage{amsmath}
\usepackage[top=1in,bottom=1in,right=1in,left=1in]{geometry}

%opening
\title{Arithematic and Geometric Sequences}
\author{Ayush Agarwal}
\date{}

\begin{document}

\maketitle

%%%%%%%%%%%%%%%%%%%%%%%%%-------------Arithematic Progression--------------%%%%%%%%%%%%%%%%%%%%%

\section{Arithematic Progression}
\subsection*{Defination}
An arithematic progression (A.P.) is a sequence of numbers such that the difference between the consecutive terms is constant. 
For instance, the sequence $5, 7, 9, 11, 13, 15 … $is an arithematic progression with common difference of 2.\\

\subsection*{Calculation}
If the initial term of an arithematic progression is $a_1$ and the common difference of successive members is d
, then the nth term of the sequence ($a_n$) is given by:\\
\vspace{2mm}
 $$a_n = a_1 + (n - 1)d,$$\\
 or, in general
 $$a_n = a_m + (n - m)d,$$\\
To derive the formula for sum of n terms($S_n$) of an A.P.\\
Let \\
\begin{eqnarray*}
S_n&=& a + (a + d) + (a + 2d) + ..... (a + (n - 1)d)\\
S_n&=&(a_n - (n - 1)d) + (a_n - (n - 2)d) + ..... + a_n - d + a_n\\
\end{eqnarray*}
\textrm Adding Both sides of the equation, we get\\
\begin{eqnarray*}
2S_n&=& n(a + a_n)\\
2S_n&=& n(a + a + (n - 1)d) \\
2S_n&=& n(2a + (n - 1)d) \\
S_n&=& \frac{n}{2}[2a + (n - 1)d] \\
\end{eqnarray*} 


%%%%%%%%%%%%%%%%%%%%%%%%%-------------Geometric Progression--------------%%%%%%%%%%%%%%%%%%%%%

\section{Geometric Progression}
\subsection*{Defination}
A Geometric progression (GP) is a sequence of numbers such that the ratio between the consecutive terms is constant. 
For instance, the sequence $2, 4, 8, 16, 32, 64 … $is an G.P. with common ratio of 2.\\
A generalized G.P. with first term \emph{a} and common ratio \emph{r} can be shown as,
$$a , ar ,  ar^2 ,  ar^3 ,  ar^4 , ....... , ar^{n-1}$$\\
with $n_{th}$ term( $a^n$) being,
$$a_n = ar^{n-1}$$\\

\subsection*{Calculation}
To derive the formula for sum of n terms($S_n$) of an G.P..\\
Let \\
\begin{align}
S_n&= a + ar + ar^2 + ..... ar^{n-1} \\
rS_n&=ar + ar^2 + ar^3 + ..... + ar^n 
\end{align}

$(1) - (2)$ , we get

\begin{align*}
S_n (1 - r) & = a - ar^n\\
S_n & = a \frac{1 - r^n}{1 - r}
\end{align*}

\end{document}
